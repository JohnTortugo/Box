% Exemplo de relatório técnico do IC
% Criado por P.J.de Rezende antes do Alvorecer da História.
% Modificado em 97-06-15 e 01-02-26 por J.Stolfi.
% Last edited on 2003-06-07 21:12:18 by stolfi

% modificado em 1o. de outubro de 2008

\documentclass[11pt,twoside]{article}
\usepackage{techrep-ic}

%%% SE USAR INGLÊS, TROQUE AS ATIVAÇÕES DOS DOIS COMANDOS A SEGUIR:
\usepackage[brazil]{babel}
%% \usepackage[english]{babel}

%%% SE USAR CODIFICAÇÃO LATIN1, TROQUE AS ATIVAÇÕES DOS DOIS COMANDOS A
%%% SEGUIR:
%% \usepackage[latin1]{inputenc}
\usepackage[utf8]{inputenc}

\begin{document}

%%% PÁGINA DE CAPA %%%%%%%%%%%%%%%%%%%%%%%%%%%%%%%%%%%%%%%%%%%%%%%
% 
% Número do relatório
\TRNumber{007}

% DATA DE PUBLICAÇÃO (PARA A CAPA)
%
\TRYear{12}  % Dois dígitos apenas
\TRMonth{09} % Numérico, 01-12

% LISTA DE AUTORES PARA CAPA (sem afiliações).
\TRAuthor{R. Barboza Jr \and A. S. Ferreira \and D. C. S. Lucas}

% TÍTULO PARA A CAPA (use \\ para forçar quebras de linha).
\TRTitle{Box \\An IA-32 Process Emulator}

\TRMakeCover

%%%%%%%%%%%%%%%%%%%%%%%%%%%%%%%%%%%%%%%%%%%%%%%%%%%%%%%%%%%%%%%%%%%%%%
% O que segue é apenas uma sugestão - sinta-se à vontade para
% usar seu formato predileto, desde que as margens tenham pelo
% menos 25mm nos quatro lados, e o tamanho do fonte seja pelo menos
% 11pt. Certifique-se também de que o título e lista de autores
% estão reproduzidos na íntegra na página 1, a primeira depois da
% página de capa.
%%%%%%%%%%%%%%%%%%%%%%%%%%%%%%%%%%%%%%%%%%%%%%%%%%%%%%%%%%%%%%%%%%%%%%

%%%%%%%%%%%%%%%%%%%%%%%%%%%%%%%%%%%%%%%%%%%%%%%%%%%%%%%%%%%%%%%%%%%%%%
% Nomes de autores ABREVIADOS e titulo ABREVIADO,
% para cabeçalhos em cada página.
%
\markboth{Barboza, Ferreira e Lucas}{Box}
\pagestyle{myheadings}

%%%%%%%%%%%%%%%%%%%%%%%%%%%%%%%%%%%%%%%%%%%%%%%%%%%%%%%%%%%%%%%%%%%%%%
% TÍTULO e NOMES DOS AUTORES, completos, para a página 1.
% Use "\\" para quebrar linhas, "\and" para separar autores.
%
\title{Box: An IA-32 Process Emulator}

\author{Roberto Barboza Jr\thanks{RA: 035712, rbarboza@gmail.com} \and
Anderson Soares Ferreira\thanks{RA: 974530, asferreira.ferreira@gmail.com} \and
Divino C. S. Lucas\thanks{RA: 115121, divcesar@gmail.com}}

\date{}

\maketitle

%%%%%%%%%%%%%%%%%%%%%%%%%%%%%%%%%%%%%%%%%%%%%%%%%%%%%%%%%%%%%%%%%%%%%%

\begin{abstract} 
  Este trabalho se trata da comparação de desempenho entre uma nova máquina 
  virtual de processos (chamda em{Box}) em relação ao em{Bochs} uma máquina
  virtual de sistema de código aberto.

  Para tal, .......
  
\end{abstract}

\section{Introdução}
% aqui vamos dar introdução ao assunto de máquinas virtuais
% falar sobre os modos de emulação 
% falar que máquinas virtuais podem ser usadas para pesquisa
% pois permitem fazer analise do código, etc.
% comentar brevemente qual ẽ o nosso objetivo no projeto.

Uma Máquina Virtual (MV) é uma plataforma versátil que pode ser empregada 
para resolver diversos problemas na área de computação. Uma MV pode ser 
visualizada como uma camada de software utilizada para prover a integração
entre duas interfaces distintas. Essas interfaces podem representar tanto
Application Binary Interface (ABI) quanto Instruction Set Architecture (ISA) 
diferentes, o que denominamos MV de Processo ou de Sistema, respectivamente. 
Nesse sentido uma MV ocupa uma posição estratégica que pode ser explorada de 
diversas formas. Por exemplo, uma MV pode efetuar a otimização dinâmica de 
binarios.

Duas abordagens são frequentemente empregadas para implementar a emulação
em uma máquina virtual: interpretação e tradução. Uma MV que emprega 
interpretação utiliza funções para simular o comportamento de cada instrução
da aplicação sendo emulada. O processo de tradução porém, emprega técnicas
da área de compiladores para produzir um código binário nativo (possívelmente
otimizado) equivalente aquele da aplicação sendo emulada. É comum encontrarmos 
MV que, visando maximizar o desempenho, empregam estas abordagens em conjunto.

Nesse texto descrevemos uma proposta de implementação de uma máquina virtual
de processo que emprega interpretação como forma de emulação. O objetivo desta
MA não é otimizar a execução da aplicação sendo emulada mas sim, criar uma
infraestrutura que permita a realização de experimentos voltados para a área
de máquinas virtuais.



   
\section{Objetivos}
% Aqui vamos apresentar nosso objetivo principal que é desenvolver
% uma máquina virtual de processo que emprega interpretação para
% fazer a emulação do sistema.
% 
% precisamos também adicionar como objetivo a análise do desempenho
% da nossa implementação em relação ao Bochs.

Como comentamos na seção anterior, o objetivo deste texto é apresentar
uma proposta de desenvolvimento de uma máquina virtual de processo que
usa interpretação como técnica de emulação. No que segue apresentamos
nosso objetivos gerais (chamaremos de étapa) e específicos (chamaremos
de passo):

\paragraph{Objetivo 1} Nesta étapa implementaremos a versão básica da 
máquina virtual. Ao final desta étapa teremos uma máquina virtual de
processo capaz de interpretar um executável compilado com linkagem
dinamica para execução em uma máquina Linux de 32 bits.
	\begin{itemize}
		\item \textbf{Carregador} Neste passo implementaremos o código responsável
		por ler e decodificar o executável principal e suas dependencias
		e montar o espaço de endereçamento onde a interpretação ocorrerá.
		\item \textbf{Decodificador} Neste passo implementaremos as rotinas de 
		decodificação das instruções x86 32 bits.
		\item \textbf{Interpretador} Neste passo implementaremos as rotinas 
		resposáveis por interpretar as instruções do x86 32 bits.
		\item \textbf{Syscalls} Neste passo implementaremos as rotinas que simularam
		as syscalls da ABI guest.
	\end{itemize}

\paragraph{Objetivo 2} Nesta étapa implementaremos os passos que adicionaram
ao sistema inicial recursos de instrumentação e otimização da interpretação.
	\begin{itemize}
		\item \textbf{Implementações Adicionais} Neste passo implementaremos uma das
		das seguintes técnicas: DICache, Threaded-Interpretation ou Pre-decoding.
		\item \textbf{Instrumentação} Neste passo implementaremos suporte para 
		instrumentação dinâmica das aplicações emuladas.
	\end{itemize}
 
\paragraph{Objetivo 3} Nesta étapa conduziremos uma análise de desempenho do sistema
em relação ao emulador Bochs e quantificaremos o impacto das técnicas implementadas
na étapa 2 no sistema.
	\begin{itemize}
		\item \textbf{Desempenho 1} Neste passo avaliaremos o desempenho do sistema
		em relação ao emulador de sistema Bochs.
		\item \textbf{Desempenho 2} Neste passo iremos avaliar o efeito das técnicas 
		implementadas na étapa 2 no desempenho do sistema.
	\end{itemize}

Na próxima seção apresentamos os trabalhos relacionados a este projeto.


\section{Levantamento Bibliográfico}
% Aqui devemos comentar sobre os trabalhos relacionados ao nosso
% podemos comentar sobre o Bochs, o PIN e o HDTrans por exemplo.
% comentar no que nosso sistema é diferente do deles.


\section{Infraestrutura}
% Aqui explicamos que estamos reaproveitando as rotinas de interpretação
% utilizadas no Bochs.
% talvez seja interessante comentar brevemente que usamos o GitHub
% para sincronizar o desenvolvimento do projeto.
%


\section{Metodologia e Cronograma}
% Aqui explicamos como vamos fazer para desenvolver o projeto,
% quais são as principais etapas do projeto e como elas estão
% distribuidas no intervalo de tempo que temos.
% precisamos incluir no cronograma a redação do relatório final
% e do relatório atual.
%


\section{Resultados e Impacto Esperado}
% Aqui falamos do que esperamos obter com o desenvolvimento do 
% projeto. Um dos resultados que pretendemos obter é análise da
% sobrecarga da emulação do sistema operacional na emulação do
% programa. 
%


\begin{thebibliography}{99}

\bibitem{AHU} A. V. Aho, J. E. Hopcroft and J.  D.  Ullman, {\it The
Design and Analysis of Computer Algorithms,} Addison-Wesley (1901).

\bibitem{KNU} D. E. Knuth and L. Lamport, {\it A structural analysis
of the role of gnus and gnats in the post-modernistic, crypto-existential 
Weltanschauung of neo-liberal Tibeto-Vietnamese leaf blower operators 
as manifest in the sexual symbology of the Los Angeles Phone Directory}.
Journal of Gnu Technology, {\bf 23} (6), 12--87
(March 1996).

\end{thebibliography}

\end{document}
