% Exemplo de relatório técnico do IC
% Criado por P.J.de Rezende antes do Alvorecer da História.
% Modificado em 97-06-15 e 01-02-26 por J.Stolfi.
% Last edited on 2003-06-07 21:12:18 by stolfi

% modificado em 1o. de outubro de 2008

\documentclass[11pt,twoside]{article}
\usepackage{techrep-ic}

%%% SE USAR INGLÊS, TROQUE AS ATIVAÇÕES DOS DOIS COMANDOS A SEGUIR:
\usepackage[brazil]{babel}
%% \usepackage[english]{babel}

%%% SE USAR CODIFICAÇÃO LATIN1, TROQUE AS ATIVAÇÕES DOS DOIS COMANDOS A
%%% SEGUIR:
%% \usepackage[latin1]{inputenc}
\usepackage[utf8]{inputenc}

\begin{document}

%%% PÁGINA DE CAPA %%%%%%%%%%%%%%%%%%%%%%%%%%%%%%%%%%%%%%%%%%%%%%%
% 
% Número do relatório
\TRNumber{007}

% DATA DE PUBLICAÇÃO (PARA A CAPA)
%
\TRYear{12}  % Dois dígitos apenas
\TRMonth{09} % Numérico, 01-12

% LISTA DE AUTORES PARA CAPA (sem afiliações).
\TRAuthor{R. Barboza Jr \and A. S. Ferreira \and D. C. S. Lucas}

% TÍTULO PARA A CAPA (use \\ para forçar quebras de linha).
\TRTitle{Box \\An IA-32 Process Emulator}

\TRMakeCover

%%%%%%%%%%%%%%%%%%%%%%%%%%%%%%%%%%%%%%%%%%%%%%%%%%%%%%%%%%%%%%%%%%%%%%
% O que segue é apenas uma sugestão - sinta-se à vontade para
% usar seu formato predileto, desde que as margens tenham pelo
% menos 25mm nos quatro lados, e o tamanho do fonte seja pelo menos
% 11pt. Certifique-se também de que o título e lista de autores
% estão reproduzidos na íntegra na página 1, a primeira depois da
% página de capa.
%%%%%%%%%%%%%%%%%%%%%%%%%%%%%%%%%%%%%%%%%%%%%%%%%%%%%%%%%%%%%%%%%%%%%%

%%%%%%%%%%%%%%%%%%%%%%%%%%%%%%%%%%%%%%%%%%%%%%%%%%%%%%%%%%%%%%%%%%%%%%
% Nomes de autores ABREVIADOS e titulo ABREVIADO,
% para cabeçalhos em cada página.
%
\markboth{Barboza, Ferreira e Lucas}{Box}
\pagestyle{myheadings}

%%%%%%%%%%%%%%%%%%%%%%%%%%%%%%%%%%%%%%%%%%%%%%%%%%%%%%%%%%%%%%%%%%%%%%
% TÍTULO e NOMES DOS AUTORES, completos, para a página 1.
% Use "\\" para quebrar linhas, "\and" para separar autores.
%
\title{Box: An IA-32 Process Emulator}

\author{Roberto Barboza Jr\thanks{RA: 035712, rbarboza@gmail.com} \and
Anderson Soares Ferreira\thanks{RA: 974530, asferreira.ferreira@gmail.com} \and
Divino C. S. Lucas\thanks{RA: 115121, divcesar@gmail.com}}

\date{}

\maketitle

%%%%%%%%%%%%%%%%%%%%%%%%%%%%%%%%%%%%%%%%%%%%%%%%%%%%%%%%%%%%%%%%%%%%%%

\begin{abstract} 
  Este trabalho se trata da comparação de desempenho entre uma nova máquina 
  virtual de processos (chamda em{Box}) em relação ao em{Bochs} uma máquina
  virtual de sistema de código aberto.

  Para tal, .......
  
\end{abstract}

\section{Introdução}
% aqui vamos dar introdução ao assunto de máquinas virtuais
% falar sobre os modos de emulação 
% falar que máquinas virtuais podem ser usadas para pesquisa
% pois permitem fazer analise do código, etc.
% comentar brevemente qual ẽ o nosso objetivo no projeto.

Uma Máquina Virtual (MV) é uma plataforma versátil que pode ser empregada 
para resolver diversos problemas na área de computação. Uma MV pode ser 
visualizada como uma camada de software utilizada para prover a integração
entre duas interfaces distintas. Essas interfaces podem representar tanto
Application Binary Interface (ABI) quanto Instruction Set Architecture (ISA) 
diferentes, o que denominamos MV de Processo ou de Sistema, respectivamente. 
Nesse sentido uma MV ocupa uma posição estratégica que pode ser explorada de 
diversas formas. Por exemplo, uma MV pode efetuar a otimização dinâmica de 
binarios.

Duas abordagens são frequentemente empregadas para implementar a emulação
em uma máquina virtual: interpretação e tradução. Uma MV que emprega 
interpretação utiliza funções para simular o comportamento de cada instrução
da aplicação sendo emulada. O processo de tradução porém, emprega técnicas
da área de compiladores para produzir um código binário nativo (possívelmente
otimizado) equivalente aquele da aplicação sendo emulada. É comum encontrarmos 
MV que, visando maximizar o desempenho, empregam estas abordagens em conjunto.

Nesse texto descrevemos uma proposta de implementação de uma máquina virtual
de processo que emprega interpretação como forma de emulação. O objetivo desta
MA não é otimizar a execução da aplicação sendo emulada mas sim, criar uma
infraestrutura que permita a realização de experimentos voltados para a área
de máquinas virtuais.



   
\section{Objetivos}
% Aqui vamos apresentar nosso objetivo principal que é desenvolver
% uma máquina virtual de processo que emprega interpretação para
% fazer a emulação do sistema.
% 
% precisamos também adicionar como objetivo a análise do desempenho
% da nossa implementação em relação ao Bochs.

Como comentamos na seção anterior, o objetivo deste texto é apresentar
uma proposta de desenvolvimento de uma máquina virtual de processo que
usa interpretação como técnica de emulação. No que segue apresentamos
nosso objetivos gerais (chamaremos de etapa) e específicos (chamaremos
de passo):

\paragraph{Primeira Etapa} Nesta etapa implementaremos a versão básica da 
máquina virtual. Ao final desta etapa teremos uma máquina virtual de
processo capaz de interpretar um executável compilado com linkagem
dinamica para execução em uma máquina Linux de 32 bits.
	\begin{itemize}
		\item \textbf{Carregador} Neste passo implementaremos o código responsável
		por ler e decodificar o executável principal e suas dependencias
		e montar o espaço de endereçamento onde a interpretação ocorrerá.
		\item \textbf{Decodificador} Neste passo implementaremos as rotinas de 
		decodificação das instruções x86 32 bits.
		\item \textbf{Interpretador} Neste passo implementaremos as rotinas 
		resposáveis por interpretar as instruções do x86 32 bits.
		\item \textbf{Syscalls} Neste passo implementaremos as rotinas que simularam
		as syscalls da ABI guest.
	\end{itemize}

\paragraph{Segunda Etapa} Nesta etapa implementaremos os passos que adicionam
ao sistema inicial recursos de instrumentação e otimização da interpretação.
	\begin{itemize}
		\item \textbf{Otimização} Neste passo implementaremos uma das
		das seguintes técnicas: DICache, Threaded-Interpretation ou Pre-decoding.
		\item \textbf{Instrumentação} Neste passo implementaremos suporte para 
		instrumentação dinâmica das aplicações emuladas.
	\end{itemize}
 
\paragraph{Terceira Etapa} Nesta etapa conduziremos uma análise de desempenho do sistema
em relação ao emulador Bochs e quantificaremos o impacto das técnicas implementadas
na etapa 2 no sistema.
	\begin{itemize}
		\item \textbf{Desempenho 1} Neste passo avaliaremos o desempenho do sistema
		em relação ao emulador de sistema Bochs.
		\item \textbf{Desempenho 2} Neste passo iremos avaliar o efeito das técnicas 
		implementadas na etapa 2 no desempenho do sistema.
	\end{itemize}

Na próxima seção apresentamos os trabalhos relacionados a este projeto.


\section{Levantamento Bibliográfico}
% Aqui devemos comentar sobre os trabalhos relacionados ao nosso
% podemos comentar sobre o Bochs, o PIN e o HDTrans por exemplo.
% comentar no que nosso sistema é diferente do deles.

\subsection{Máquinas Virtuais}
O Bochs \cite{bochs} é uma máquina virtual de sistema com suporte a emulação
de diversos processadores da família x86 32 e 64 bits. Ele preza
por portabilidade, e nesse sentido utiliza interpretação como técnica
de emulação. Para reduzir o overhead de interpretação, o Bochs utiliza
as técnicas de pre-decoding \cite{smith} e threaded-interpretation \cite{smith}.
O sistema que estamos propondo se diferencia do Bochs por ser voltado
à emulação de processos e não de sistemas.

O Pin \cite{pin} é uma máquina virtual de processo que emula os ISAs IA-32
e x86-64. Ele é uma ferramenta de código fechado e possui versões tanto
para Windows como para Linux, sendo reconhecido por prover uma rica
API para criação de ferramentas (Pintools) para análise do comportamento 
dinâmico de aplicações. O Box diferencia-se do PIN por ser uma máquina
virtual de código aberto e utilizar interpretação como técnica de emulação.
Em contrapartida, os sistemas se assemelham no aspecto de proverem uma
interface para instrumentação de binários e serem emuladores same-isa~\footnote{As duas interfaces da máquina virtual são iguais.}.

O HDTrans \cite{hdtrans} é uma máquina virtual de processo para
a arquitetura IA-32. O HDTrans difere do Bochs e do PIN por empregar tradução
de binários como técnica de emulação. No entanto, o HDTrans abre mão de 
empregar otimizações no código traduzido (o que geralmente é feito para 
amortizar o overhead de tradução) em favor da implementação de uma tradução
simples e rápida. O Box diferencia-se do HDTrans por empregar interpretação 
de binários.

O Dynamo \cite{dynamo} é uma máquina virtual de processo para a arquitetura
do HP PA-800. Diferentemente do PIN, Bochs e HDTrans, o Dynamo é uma máquina
virtual cujo objetivo é otimizar a execução do programa sendo emulado.
Para tanto o Dynamo emprega uma combinação das técnicas de interpretação
e emulação. O sistema inicialmente interpreta o código da aplicação sendo 
emulada. Uma vez que uma região é declarada como quente, o Dynamo aplica 
otimizações nesta região e salva o código otimizado em uma cache de traduções. 
Subsequentes invocações do trecho de código original serão emuladas utilizando 
o código traduzido. O Box diferencia-se do Dynamo por não ter o foco na otimização 
de binários mas sim em ser um sistema que prove uma interface simples para 
instrumentação de binários.

O IA-32 EL \cite{ia32el} é uma máquina virtual de processos com o objetivo de
suportar a execução de aplicações IA-32 em processadores da família IA-64.
De forma similar ao Dynamo, o IA-32 EL emprega uma abordagem de emulação em
duas etapas, porém as duas etapas realizam tradução de código. Inicialmente
o IA-32 EL efetua a tradução de código em uma granularidade de bloco básico.
Durante essa tradução, código de instrumentação é inserido para detectar blocos
básicos que são frequentemente executados. Quando um número mínimo de blocos básicos
é identificado, o sistema forma uma região de código envolvendo esses blocos
básicos, os otimiza e posteriormente salva em uma cache de traduções. Execuções
subsequentes desses blocos básicos usam a tradução otimizada. O Box diferencia-se 
do IA-32 EL por não fazer tradução de binários.

StarDBT \cite{stardbt} é uma máquina virtual de pesquisa capaz de fazer traduções de
aplicações compiladas para x86 32/64 bits para execução em hardware x86 32 bits.
De forma similar ao IA-32 EL, o StarDBT é um tradutor em duas fases. Regiões de
código que são infrequentemente executadas são traduzidas utilizando um tradutor
simples e rápido, enquanto regiões de código que são frequentemente executadas
são traduzidas empregando otimizações de código e posteriormente persistidas
em uma cache de traduções. O Box diferencia-se do StarDBT por ser capaz de traduzir
apenas binários de 32 bits, compilados para Linux e não empregar tradução
de binários.

% acho que esse nome não está bom...
\section{Técnicas de Interpretação}


% acham que precisamos incluir esta subsecao?
% penso em colocar aqui um comentário sobre o ELF e sobre as syscalls do linux
\section{Application Binary Interface}


\section{Infraestrutura de Pesquisa}
% Aqui explicamos que estamos reaproveitando as rotinas de interpretação
% utilizadas no Bochs.
% talvez seja interessante comentar brevemente que usamos o GitHub
% para sincronizar o desenvolvimento do projeto.
%


\section{Metodologia e Cronograma}
% Aqui explicamos como vamos fazer para desenvolver o projeto,
% quais são as principais etapas do projeto e como elas estão
% distribuidas no intervalo de tempo que temos.
% precisamos incluir no cronograma a redação do relatório final
% e do relatório atual.
%

O sistema que estamos propondo utilizará o Bochs como ponto de partida
para desenvolvimento. Uma vez que o Bochs é uma máquina virtual de sistema,
ele emula o comportamento de diversos dispositivos de hardware, como
cache de instruções, placa de vídeo, placa de rede, etc. Como o sistema
que estamos projetando é uma máquina virtual de processo não precisamos
emular tais dispositivos. 

\begin{thebibliography}{99}

\bibitem{pin}Chi-Keung Luk, Robert Cohn, Robert Muth, Harish Patil, Artur
Klauser, Geoff Lowney, Steven Wallace, Vijay Janapa Reddi, Kim Hazelwood, 
Pin: building customized program analysis tools with dynamic instrumentation, 
Proceedings of the 2005 ACM SIGPLAN conference on Programming language design
and implementation, June 12-15, 2005, Chicago, IL, USA 

\bibitem{hdtrans}Swaroop Sridhar, Jonathan S. Shapiro, Prashanth P. Bungale, 
HDTrans: a low-overhead dynamic translator, ACM SIGARCH Computer Architecture
News, v.35 n.1, p.135-140, March 2007 

\bibitem{dynamo}Bala, V., Duesterwald, E., and Banerjia, S. 1999. Transparent
dynamic optimization: The design and implementation of Dynamo. Hewlett Packard
Laboratories Technical Report HPL-1999-78. June 1999. 

\end{thebibliography}

\end{document}
